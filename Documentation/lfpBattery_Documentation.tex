\documentclass [a4paper, 12pt] {scrartcl}		%Textformat Article, KomaScript
\linespread{1.2}								% Setzt Zeilenabstand auf x*1.2 (KomaScript)
\renewcommand{\familydefault}{\sfdefault}		% Setzt Schrift auf Serifenlos (fast wie Arial)


\usepackage{amsfonts} %   AMS sind für den Mathematikmodus
\usepackage{amsmath}
\usepackage{acronym}
\usepackage{amssymb}
\usepackage{longtable}
\usepackage[hidelinks]{hyperref} % Paket für Hyperlinks
\usepackage{tabularx} % für andere Darstellung von Tabellen z.B. mit dickem Rahmen
\usepackage{graphicx, color} % zum Einbinden von Grafiken
\usepackage{fancyhdr}	% Paket für Kopf- und Fußzeile
\usepackage[margin=1in,headsep=.6in]{geometry} % Für Abstand zwischen Kopfzeile & Text
\usepackage{rotating}	% Bilder per rotate umgebung drehen
\usepackage{wrapfig} 	% Text um Bild fließen lassen
\usepackage{multirow}	% Text über mehrere Spalten und Zeilen in Tabelle
\usepackage{textcomp}	% Copyleftzeichen
\usepackage[utf8]{inputenc} %  UFT8 Zeichenkodierung
\usepackage{color}	% farbige Elemnete, zum Beispiel Rahmen, Linien etc.
\usepackage{caption} % falls man Bildunterschriten und Tabellenüberschriften formatieren will
%\usepackage[ngerman]{babel} %deutsche Rechtschreibung
\usepackage{booktabs}
\usepackage{epstopdf}
\usepackage[xindy,toc,nonumberlist,toc,section]{glossaries}
\usepackage{blindtext}
\usepackage{microtype}
\usepackage{color,siunitx} %for SI-units
\usepackage{todonotes}

\usepackage{subfig} %subfig statt subfigure (das veraltet ist)
\usepackage{csquotes}
\usepackage{perpage} % perpage-Paket (resettet vorgegebene Zähler für jede neue Seite)
\MakePerPage{footnote} % wendet perpage-Paket auf Fußnoten an

\usepackage[backend=biber,natbib=true,hyperref=true,style=ieee]{biblatex} %Biblatex mit biber (IEEE) [style=ieee,backend=biber,natbib=true,hyperref=true]
\addbibresource{PRO.bib}
%
% Eigene definierte Zeichen
\newcommand{\sei}{\stackrel{!}{=}} % Soll-Gleich-Sein Zeichen
\newcommand{\cel}{$^{\circ}$C \ } % Grad Celsius in Textumgebung
\newcommand{\mcel}{^{\circ}C \ } % Gra Celsius und Mathematikumgebung
\newcommand{\matlab}{Matlab\textsuperscript{\textregistered}} % Matlab mit (R)-Zeichen
\newcommand{\polysun}{Polysun\textsuperscript{\textregistered}} % Polysun mit (R)-Zeichen
\newcommand{\java}{JAVA\textsuperscript{\texttrademark}} %  JAVA mit TM-Zeichen
\newcommand{\subi}[1]{_\text{#1}} % subscript index
\newcommand{\subs}[2]{_{\text{#1}, #2}} % subscript index + Laufindex
\newcommand{\dexp}[1]{\cdot 10^{#1}} % *10^...

%Definition der HTW Farben für Überschriften usw.
\usepackage{xcolor}
\definecolor{HTWgreen}{RGB}{119,185,0}
\definecolor{grey}{RGB}{175,175,175}
\definecolor{TUred}{RGB}{197,14,31}
\definecolor{TUgrey}{RGB}{113,113,113}
\usepackage{sectsty}
\chapterfont{\color{TUred}}  % sets colour of chapters
\sectionfont{\color{TUred}}  % sets colour of sections
\subsectionfont{\color{TUgrey}}  % sets colour of sections

\pagestyle{fancy} % leere unformatierte Seite
\renewcommand{\sectionmark}[1]{\markright{\thesection.\ #1}}

%Kopfzeile definieren
\lhead{\includegraphics[width=0.1\textwidth]{tulogo.pdf}} %  HTW Logo in linke Kopfzeile
%\chead{\leftmark} % mittlere Kopfzeile
\rhead{\hspace{-0.14in}\vspace{-0.14in}\includegraphics[width=0.25\textwidth]{htwlogo}} % rechte Kopfzeile
\chead{}
%\rhead{\leftmark}

%Fußzeile definieren
\lfoot{} %links
\cfoot{\thepage} % Mitte mit Anzeige der Seitenzahl
\rfoot{} % rechts 
\renewcommand{\headrulewidth}{0.4pt}	%Dicke der Kopf und Fußzeilentrenner
\renewcommand{\footrulewidth}{0.4pt} 

\captionsetup{format=hang,font=small,labelfont=bf,textfont=it,
	justification=justified,singlelinecheck=false}
\renewcommand{\thefootnote}{\roman{footnote}} % Fußnoten in römischen Zeichen

\begin{document}
	
\begin{titlepage}
	\begin{figure}
%		\includegraphics[width=\textwidth]{HTW-Titel.png}
			\vskip 30px % Abstand zum Runterschieben
		\begin{minipage}[b]{\textwidth}
			\flushright % oder \flushleft oder \center, wie es halt ausgerichtet sein soll
			{\LARGE\textbf{Cell Resolved \matlab\ OOP Model of a Lithium Iron Phosphate Battery Pack}}\\
			\vskip 10px
			\textbf{Marc Jakobi, Festus Anyangbe, Marc Schmitdt} \\
			\today \\ % current date
			HTW Berlin \\
			\vskip 10px
			Supervision: \\ \textbf{M.Sc. Steven Neupert}\\
			TU Berlin \\
			\vskip 10px
		\end{minipage}
	\end{figure}
\end{titlepage}
\clearpage

\begin{titlepage}
	\fancypagestyle{myheadings}{Inhalt}
	\tableofcontents
	
\end{titlepage}\clearpage
\setcounter{page}{1}

\pagenumbering{roman}

\listoffigures
\listoftables
\section*{Abbreviations}
\thispagestyle{plain}
\markboth{Abbreviations}{Abbreviations}
\begin{acronym}
	\acro{BMS}[BMS] battery management system
	\acro{MEX}[MEX] \matlab\ executable
	\acro{OO}[OO] object oriented
	\acro{OOP}[OOP] object oriented programming
	
\end{acronym}
\section*{List of Symbols}
%\addcontentsline{toc}{chapter}{Formelverzeichnis} 	%Einbinden Des Verzeichnisses in das Inhaltsverzeichnis
\thispagestyle{plain}	%erste Seite des Kapitels ohne Kopfzeile
\markboth{List of symbols}{List of symbols}
\captionsetup{list=false}%captions werden nicht in die zugehörigen Verzeichnisse geschrieben

\begin{longtable}{lrl}
\captionlistentry{Symbol}\\
\toprule
Symbol		 					& Unit  	& Description \\
\midrule
$C\subi{dis}$					& Ah						& discharge capacity \\
$F$								& As/mol					& Faraday constant \\
$I$								& A							& Current \\
$SoC$							& \%						& state of charge \\
$R$								& J/(mol $\cdot$ K)			& universal gas constant  \\
$rmse$							& -							& root mean squared error \\
$T$								& K or \cel					& temperature \\
$V$								& V							& voltage \\
$z\subi{Li}$					& -							& ionic charge number of lithium \\
\bottomrule
\end{longtable}

\captionsetup{list=true}%captions werden in die zugehörigen Verzeichnisse geschrieben
\setcounter{table}{0}


\captionsetup{list=true} %captions werden in die zugehörigen Verzeichnisse geschrieben
\setcounter{table}{0}
\clearpage
\setcounter{page}{1}
\pagenumbering{arabic}


%% INCLUDES / SUBSECTIONS HERE

\addcontentsline{toc}{section}{References}
\printbibliography

\end{document}